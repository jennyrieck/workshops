\documentclass[]{article}
\usepackage{lmodern}
\usepackage{amssymb,amsmath}
\usepackage{ifxetex,ifluatex}
\usepackage{fixltx2e} % provides \textsubscript
\ifnum 0\ifxetex 1\fi\ifluatex 1\fi=0 % if pdftex
  \usepackage[T1]{fontenc}
  \usepackage[utf8]{inputenc}
\else % if luatex or xelatex
  \ifxetex
    \usepackage{mathspec}
  \else
    \usepackage{fontspec}
  \fi
  \defaultfontfeatures{Ligatures=TeX,Scale=MatchLowercase}
\fi
% use upquote if available, for straight quotes in verbatim environments
\IfFileExists{upquote.sty}{\usepackage{upquote}}{}
% use microtype if available
\IfFileExists{microtype.sty}{%
\usepackage{microtype}
\UseMicrotypeSet[protrusion]{basicmath} % disable protrusion for tt fonts
}{}
\usepackage[margin=1in]{geometry}
\usepackage{hyperref}
\PassOptionsToPackage{usenames,dvipsnames}{color} % color is loaded by hyperref
\hypersetup{unicode=true,
            pdftitle={Simple Markdown Example},
            pdfauthor={Jenny Rieck; Derek Beaton},
            colorlinks=true,
            linkcolor=Maroon,
            citecolor=Blue,
            urlcolor=blue,
            breaklinks=true}
\urlstyle{same}  % don't use monospace font for urls
\usepackage{color}
\usepackage{fancyvrb}
\newcommand{\VerbBar}{|}
\newcommand{\VERB}{\Verb[commandchars=\\\{\}]}
\DefineVerbatimEnvironment{Highlighting}{Verbatim}{commandchars=\\\{\}}
% Add ',fontsize=\small' for more characters per line
\usepackage{framed}
\definecolor{shadecolor}{RGB}{248,248,248}
\newenvironment{Shaded}{\begin{snugshade}}{\end{snugshade}}
\newcommand{\AlertTok}[1]{\textcolor[rgb]{0.94,0.16,0.16}{#1}}
\newcommand{\AnnotationTok}[1]{\textcolor[rgb]{0.56,0.35,0.01}{\textbf{\textit{#1}}}}
\newcommand{\AttributeTok}[1]{\textcolor[rgb]{0.77,0.63,0.00}{#1}}
\newcommand{\BaseNTok}[1]{\textcolor[rgb]{0.00,0.00,0.81}{#1}}
\newcommand{\BuiltInTok}[1]{#1}
\newcommand{\CharTok}[1]{\textcolor[rgb]{0.31,0.60,0.02}{#1}}
\newcommand{\CommentTok}[1]{\textcolor[rgb]{0.56,0.35,0.01}{\textit{#1}}}
\newcommand{\CommentVarTok}[1]{\textcolor[rgb]{0.56,0.35,0.01}{\textbf{\textit{#1}}}}
\newcommand{\ConstantTok}[1]{\textcolor[rgb]{0.00,0.00,0.00}{#1}}
\newcommand{\ControlFlowTok}[1]{\textcolor[rgb]{0.13,0.29,0.53}{\textbf{#1}}}
\newcommand{\DataTypeTok}[1]{\textcolor[rgb]{0.13,0.29,0.53}{#1}}
\newcommand{\DecValTok}[1]{\textcolor[rgb]{0.00,0.00,0.81}{#1}}
\newcommand{\DocumentationTok}[1]{\textcolor[rgb]{0.56,0.35,0.01}{\textbf{\textit{#1}}}}
\newcommand{\ErrorTok}[1]{\textcolor[rgb]{0.64,0.00,0.00}{\textbf{#1}}}
\newcommand{\ExtensionTok}[1]{#1}
\newcommand{\FloatTok}[1]{\textcolor[rgb]{0.00,0.00,0.81}{#1}}
\newcommand{\FunctionTok}[1]{\textcolor[rgb]{0.00,0.00,0.00}{#1}}
\newcommand{\ImportTok}[1]{#1}
\newcommand{\InformationTok}[1]{\textcolor[rgb]{0.56,0.35,0.01}{\textbf{\textit{#1}}}}
\newcommand{\KeywordTok}[1]{\textcolor[rgb]{0.13,0.29,0.53}{\textbf{#1}}}
\newcommand{\NormalTok}[1]{#1}
\newcommand{\OperatorTok}[1]{\textcolor[rgb]{0.81,0.36,0.00}{\textbf{#1}}}
\newcommand{\OtherTok}[1]{\textcolor[rgb]{0.56,0.35,0.01}{#1}}
\newcommand{\PreprocessorTok}[1]{\textcolor[rgb]{0.56,0.35,0.01}{\textit{#1}}}
\newcommand{\RegionMarkerTok}[1]{#1}
\newcommand{\SpecialCharTok}[1]{\textcolor[rgb]{0.00,0.00,0.00}{#1}}
\newcommand{\SpecialStringTok}[1]{\textcolor[rgb]{0.31,0.60,0.02}{#1}}
\newcommand{\StringTok}[1]{\textcolor[rgb]{0.31,0.60,0.02}{#1}}
\newcommand{\VariableTok}[1]{\textcolor[rgb]{0.00,0.00,0.00}{#1}}
\newcommand{\VerbatimStringTok}[1]{\textcolor[rgb]{0.31,0.60,0.02}{#1}}
\newcommand{\WarningTok}[1]{\textcolor[rgb]{0.56,0.35,0.01}{\textbf{\textit{#1}}}}
\usepackage{graphicx,grffile}
\makeatletter
\def\maxwidth{\ifdim\Gin@nat@width>\linewidth\linewidth\else\Gin@nat@width\fi}
\def\maxheight{\ifdim\Gin@nat@height>\textheight\textheight\else\Gin@nat@height\fi}
\makeatother
% Scale images if necessary, so that they will not overflow the page
% margins by default, and it is still possible to overwrite the defaults
% using explicit options in \includegraphics[width, height, ...]{}
\setkeys{Gin}{width=\maxwidth,height=\maxheight,keepaspectratio}
\IfFileExists{parskip.sty}{%
\usepackage{parskip}
}{% else
\setlength{\parindent}{0pt}
\setlength{\parskip}{6pt plus 2pt minus 1pt}
}
\setlength{\emergencystretch}{3em}  % prevent overfull lines
\providecommand{\tightlist}{%
  \setlength{\itemsep}{0pt}\setlength{\parskip}{0pt}}
\setcounter{secnumdepth}{0}
% Redefines (sub)paragraphs to behave more like sections
\ifx\paragraph\undefined\else
\let\oldparagraph\paragraph
\renewcommand{\paragraph}[1]{\oldparagraph{#1}\mbox{}}
\fi
\ifx\subparagraph\undefined\else
\let\oldsubparagraph\subparagraph
\renewcommand{\subparagraph}[1]{\oldsubparagraph{#1}\mbox{}}
\fi

%%% Use protect on footnotes to avoid problems with footnotes in titles
\let\rmarkdownfootnote\footnote%
\def\footnote{\protect\rmarkdownfootnote}

%%% Change title format to be more compact
\usepackage{titling}

% Create subtitle command for use in maketitle
\newcommand{\subtitle}[1]{
  \posttitle{
    \begin{center}\large#1\end{center}
    }
}

\setlength{\droptitle}{-2em}

  \title{Simple Markdown Example}
    \pretitle{\vspace{\droptitle}\centering\huge}
  \posttitle{\par}
  \subtitle{PDF/LaTex version}
  \author{Jenny Rieck \\ Derek Beaton}
    \preauthor{\centering\large\emph}
  \postauthor{\par}
      \predate{\centering\large\emph}
  \postdate{\par}
    \date{May 12, 2019}

\usepackage{booktabs}
\usepackage{longtable}
\usepackage{array}
\usepackage{multirow}
\usepackage[table]{xcolor}
\usepackage{wrapfig}
\usepackage{float}
\usepackage{colortbl}
\usepackage{pdflscape}
\usepackage{tabu}
\usepackage{threeparttable}
\usepackage{threeparttablex}
\usepackage[normalem]{ulem}
\usepackage{makecell}

\begin{document}
\maketitle

\hypertarget{introduction}{%
\section{Introduction}\label{introduction}}

For reference, see
\href{https://emilyriederer.netlify.com/post/rmarkdown-driven-development/}{RMarkdown
Driven Development} by Emily Riederer for a comprehensive overview of
how to best structure RMarkdown for projects, packages, and other
development-driven tasks.

In general for this RMarkdown file we follow Emily's fourth example with
a directory structure, minimized redundancies, and heavy-duty code
elsewhere. This RMarkdown, generally, serves as place to describe,
analyze, and visualize our data. Thus, this text is shown at the top of
the output, but actually appears after several \texttt{R} code chunks,
which exist between the \texttt{Introduction} heading and YAML header
information (which you see as title, authors, dates, etc\ldots{}).

Most of the RMarkdown files in this directory will show generally the
same content, but help highlight the different ways you can use
RMarkdown, \texttt{knitr}, \texttt{pandoc}, \texttt{LateX}, and various
package built for those, such as \texttt{beamer} (\texttt{LateX}) for
presentations, and
\href{https://github.com/crsh/papaja}{\texttt{papaja}} \&
\href{https://github.com/rstudio/rticles}{\texttt{rticles}}
(R/RMarkdown) for writing manuscripts that export to \texttt{LaTeX} or
MS Word. If you decide to write MS Word documents through RMarkdown, you
should also use the
\href{https://github.com/noamross/redoc}{\texttt{redoc} package}.

\hypertarget{r-chunks-a-word-of-caution}{%
\subsection{R chunks: A word of
caution}\label{r-chunks-a-word-of-caution}}

It is good practice to name your \texttt{R} chunks. If you do not, then
the \texttt{R} chunks will still produce the intended material. However,
when you do name them, you should ensure they have unique names (else,
you will likely see some cryptic and not always informative error
messages).

\hypertarget{tables}{%
\section{Tables}\label{tables}}

There are multiple approaches and packages to help visualize tables or
tabular information. Let's start by looking at a simple summary of all
the continuous variables. First, we will visualize the summary table
through two methods within R: \texttt{knitr::kable} and the
\texttt{kablExtra} package, followed by \texttt{grid} and
\texttt{gridExtra}. Next, we will use the same data and illustrate what
happens when we pass it to Python through \texttt{reticulate}.

In this section we wil also show the code chunks that generate these
tables and visuals, which are embedded in the RMarkdown document.

\hypertarget{knitr-and-kableextra}{%
\subsection{knitr and kableExtra}\label{knitr-and-kableextra}}

To make HTML and LaTeX tables in RMarkdown, one of the easiest and most
common options is through \texttt{knitr}. The \texttt{knitr} package is,
effectively, the tool to make RMarkdown documents go from R \& RMarkdown
(plus other code and LaTeX) into PDFs or HTML pages. We'll start with
the \texttt{knitr::kable()}.

\begin{Shaded}
\begin{Highlighting}[]
\NormalTok{example_table <-}\StringTok{ }\NormalTok{amerge_subset[, }\KeywordTok{c}\NormalTok{(}\StringTok{"AGE"}\NormalTok{, }\StringTok{"MOCA"}\NormalTok{, }\StringTok{"CDRSB"}\NormalTok{, }\StringTok{"WholeBrain"}\NormalTok{, }
    \StringTok{"Hippocampus"}\NormalTok{, }\StringTok{"MidTemp"}\NormalTok{)]}
\NormalTok{example_table_Dx <-}\StringTok{ }\NormalTok{amerge_subset[, }\KeywordTok{c}\NormalTok{(}\StringTok{"DX"}\NormalTok{, }\StringTok{"AGE"}\NormalTok{, }\StringTok{"MOCA"}\NormalTok{, }\StringTok{"CDRSB"}\NormalTok{, }
    \StringTok{"WholeBrain"}\NormalTok{, }\StringTok{"Hippocampus"}\NormalTok{, }\StringTok{"MidTemp"}\NormalTok{)]}
\end{Highlighting}
\end{Shaded}

\begin{Shaded}
\begin{Highlighting}[]
\KeywordTok{kable}\NormalTok{(}\KeywordTok{summary}\NormalTok{(example_table))}
\end{Highlighting}
\end{Shaded}

\begin{tabular}{l|l|l|l|l|l|l}
\hline
  &      AGE &      MOCA &     CDRSB &   WholeBrain &  Hippocampus &    MidTemp\\
\hline
 & Min.   :55.00 & Min.   :16.00 & Min.   :0.000 & Min.   : 817421 & Min.   : 3731 & Min.   :12213\\
\hline
 & 1st Qu.:67.20 & 1st Qu.:22.00 & 1st Qu.:0.000 & 1st Qu.: 984410 & 1st Qu.: 6510 & 1st Qu.:18535\\
\hline
 & Median :71.90 & Median :24.00 & Median :1.000 & Median :1051621 & Median : 7223 & Median :20186\\
\hline
 & Mean   :71.92 & Mean   :23.89 & Mean   :1.202 & Mean   :1057026 & Mean   : 7150 & Mean   :20302\\
\hline
 & 3rd Qu.:76.60 & 3rd Qu.:26.00 & 3rd Qu.:2.000 & 3rd Qu.:1120570 & 3rd Qu.: 7834 & 3rd Qu.:22088\\
\hline
 & Max.   :89.60 & Max.   :30.00 & Max.   :5.500 & Max.   :1486036 & Max.   :10602 & Max.   :32189\\
\hline
\end{tabular}

But that is not particularly nice looking. So we can use some parameters
to make this table look better (which depend on having LaTeX).

\begin{Shaded}
\begin{Highlighting}[]
\KeywordTok{kable}\NormalTok{(}\KeywordTok{summary}\NormalTok{(example_table), }\DataTypeTok{format =} \StringTok{"latex"}\NormalTok{, }\DataTypeTok{booktabs =}\NormalTok{ T)}
\end{Highlighting}
\end{Shaded}

\begin{tabular}{lllllll}
\toprule
  &      AGE &      MOCA &     CDRSB &   WholeBrain &  Hippocampus &    MidTemp\\
\midrule
 & Min.   :55.00 & Min.   :16.00 & Min.   :0.000 & Min.   : 817421 & Min.   : 3731 & Min.   :12213\\
 & 1st Qu.:67.20 & 1st Qu.:22.00 & 1st Qu.:0.000 & 1st Qu.: 984410 & 1st Qu.: 6510 & 1st Qu.:18535\\
 & Median :71.90 & Median :24.00 & Median :1.000 & Median :1051621 & Median : 7223 & Median :20186\\
 & Mean   :71.92 & Mean   :23.89 & Mean   :1.202 & Mean   :1057026 & Mean   : 7150 & Mean   :20302\\
 & 3rd Qu.:76.60 & 3rd Qu.:26.00 & 3rd Qu.:2.000 & 3rd Qu.:1120570 & 3rd Qu.: 7834 & 3rd Qu.:22088\\
 & Max.   :89.60 & Max.   :30.00 & Max.   :5.500 & Max.   :1486036 & Max.   :10602 & Max.   :32189\\
\bottomrule
\end{tabular}

With \texttt{booktabs} and \texttt{latex} format, we've made the table
look a little better. But can we make it look even better than that? We
can with \texttt{kableExtra}.

\begin{Shaded}
\begin{Highlighting}[]
\KeywordTok{kable}\NormalTok{(}\KeywordTok{summary}\NormalTok{(example_table), }\DataTypeTok{format =} \StringTok{"latex"}\NormalTok{, }\DataTypeTok{booktabs =}\NormalTok{ T) }\OperatorTok\StringTok{ }
\StringTok{    }\KeywordTok{kable_styling}\NormalTok{(}\DataTypeTok{font_size =} \DecValTok{10}\NormalTok{, }\DataTypeTok{position =} \StringTok{"center"}\NormalTok{)}
\end{Highlighting}
\end{Shaded}

\begin{table}[H]
\centering\begingroup\fontsize{10}{12}\selectfont

\begin{tabular}{lllllll}
\toprule
  &      AGE &      MOCA &     CDRSB &   WholeBrain &  Hippocampus &    MidTemp\\
\midrule
 & Min.   :55.00 & Min.   :16.00 & Min.   :0.000 & Min.   : 817421 & Min.   : 3731 & Min.   :12213\\
 & 1st Qu.:67.20 & 1st Qu.:22.00 & 1st Qu.:0.000 & 1st Qu.: 984410 & 1st Qu.: 6510 & 1st Qu.:18535\\
 & Median :71.90 & Median :24.00 & Median :1.000 & Median :1051621 & Median : 7223 & Median :20186\\
 & Mean   :71.92 & Mean   :23.89 & Mean   :1.202 & Mean   :1057026 & Mean   : 7150 & Mean   :20302\\
 & 3rd Qu.:76.60 & 3rd Qu.:26.00 & 3rd Qu.:2.000 & 3rd Qu.:1120570 & 3rd Qu.: 7834 & 3rd Qu.:22088\\
 & Max.   :89.60 & Max.   :30.00 & Max.   :5.500 & Max.   :1486036 & Max.   :10602 & Max.   :32189\\
\bottomrule
\end{tabular}\endgroup{}
\end{table}

We can take the table look even further with additional options, like
``stripes''.

\begin{Shaded}
\begin{Highlighting}[]
\KeywordTok{kable}\NormalTok{(}\KeywordTok{summary}\NormalTok{(example_table), }\DataTypeTok{format =} \StringTok{"latex"}\NormalTok{, }\DataTypeTok{booktabs =}\NormalTok{ T) }\OperatorTok\StringTok{ }
\StringTok{    }\KeywordTok{kable_styling}\NormalTok{(}\DataTypeTok{font_size =} \DecValTok{10}\NormalTok{, }\DataTypeTok{position =} \StringTok{"center"}\NormalTok{, }\DataTypeTok{latex_options =} \StringTok{"striped"}\NormalTok{)}
\end{Highlighting}
\end{Shaded}

\begin{table}[H]
\centering\begingroup\fontsize{10}{12}\selectfont
\rowcolors{2}{gray!6}{white}

\begin{tabular}{lllllll}
\hiderowcolors
\toprule
  &      AGE &      MOCA &     CDRSB &   WholeBrain &  Hippocampus &    MidTemp\\
\midrule
\showrowcolors
 & Min.   :55.00 & Min.   :16.00 & Min.   :0.000 & Min.   : 817421 & Min.   : 3731 & Min.   :12213\\
 & 1st Qu.:67.20 & 1st Qu.:22.00 & 1st Qu.:0.000 & 1st Qu.: 984410 & 1st Qu.: 6510 & 1st Qu.:18535\\
 & Median :71.90 & Median :24.00 & Median :1.000 & Median :1051621 & Median : 7223 & Median :20186\\
 & Mean   :71.92 & Mean   :23.89 & Mean   :1.202 & Mean   :1057026 & Mean   : 7150 & Mean   :20302\\
 & 3rd Qu.:76.60 & 3rd Qu.:26.00 & 3rd Qu.:2.000 & 3rd Qu.:1120570 & 3rd Qu.: 7834 & 3rd Qu.:22088\\
 & Max.   :89.60 & Max.   :30.00 & Max.   :5.500 & Max.   :1486036 & Max.   :10602 & Max.   :32189\\
\bottomrule
\end{tabular}
\rowcolors{2}{white}{white}\endgroup{}
\end{table}

Given that we have redundant information in the table (min/max,
etc\ldots{}) we can do a better job and make an even nicer table with an
\texttt{apply()}, and then use multiple \texttt{kable} and
\texttt{kableExtra} features to make a really nice table.

\begin{Shaded}
\begin{Highlighting}[]
\NormalTok{better_example_table <-}\StringTok{ }\KeywordTok{apply}\NormalTok{(example_table, }\DecValTok{2}\NormalTok{, summary)}

\KeywordTok{kable}\NormalTok{(better_example_table, }\DataTypeTok{format =} \StringTok{"latex"}\NormalTok{, }\DataTypeTok{booktabs =}\NormalTok{ T, }\DataTypeTok{digits =} \DecValTok{2}\NormalTok{) }\OperatorTok\StringTok{ }
\StringTok{    }\NormalTok{kableExtra}\OperatorTok{::}\KeywordTok{add_header_above}\NormalTok{(}\KeywordTok{c}\NormalTok{(}\DataTypeTok{Statistic =} \DecValTok{1}\NormalTok{, }\DataTypeTok{Demographic =} \DecValTok{1}\NormalTok{, }
        \DataTypeTok{Clinical =} \DecValTok{2}\NormalTok{, }\DataTypeTok{Brain =} \DecValTok{3}\NormalTok{)) }\OperatorTok\StringTok{ }\KeywordTok{kable_styling}\NormalTok{(}\DataTypeTok{font_size =} \DecValTok{10}\NormalTok{, }
    \DataTypeTok{position =} \StringTok{"center"}\NormalTok{, }\DataTypeTok{latex_options =} \StringTok{"striped"}\NormalTok{) }\OperatorTok\StringTok{ }\KeywordTok{row_spec}\NormalTok{(}\DecValTok{0}\NormalTok{, }
    \DataTypeTok{angle =} \DecValTok{15}\NormalTok{, }\DataTypeTok{bold =}\NormalTok{ T)}
\end{Highlighting}
\end{Shaded}

\begin{table}[H]
\centering\begingroup\fontsize{10}{12}\selectfont
\rowcolors{2}{gray!6}{white}

\begin{tabular}{lrrrrrr}
\hiderowcolors
\toprule
\multicolumn{1}{c}{Statistic} & \multicolumn{1}{c}{Demographic} & \multicolumn{2}{c}{Clinical} & \multicolumn{3}{c}{Brain} \\
\cmidrule(l{2pt}r{2pt}){1-1} \cmidrule(l{2pt}r{2pt}){2-2} \cmidrule(l{2pt}r{2pt}){3-4} \cmidrule(l{2pt}r{2pt}){5-7}
\rotatebox{15}{\textbf{ }} & \rotatebox{15}{\textbf{AGE}} & \rotatebox{15}{\textbf{MOCA}} & \rotatebox{15}{\textbf{CDRSB}} & \rotatebox{15}{\textbf{WholeBrain}} & \rotatebox{15}{\textbf{Hippocampus}} & \rotatebox{15}{\textbf{MidTemp}}\\
\midrule
\showrowcolors
Min. & 55.00 & 16.00 & 0.0 & 817421.2 & 3731.00 & 12213.00\\
1st Qu. & 67.20 & 22.00 & 0.0 & 984409.9 & 6510.00 & 18535.00\\
Median & 71.90 & 24.00 & 1.0 & 1051621.3 & 7223.00 & 20186.00\\
Mean & 71.92 & 23.89 & 1.2 & 1057025.6 & 7149.61 & 20301.93\\
3rd Qu. & 76.60 & 26.00 & 2.0 & 1120569.5 & 7834.00 & 22088.00\\
Max. & 89.60 & 30.00 & 5.5 & 1486035.6 & 10602.00 & 32189.00\\
\bottomrule
\end{tabular}
\rowcolors{2}{white}{white}\endgroup{}
\end{table}

\hypertarget{grid-and-gridextra}{%
\subsection{grid and gridExtra}\label{grid-and-gridextra}}

\hypertarget{python-via-reticulate}{%
\subsection{Python via reticulate}\label{python-via-reticulate}}


\end{document}
